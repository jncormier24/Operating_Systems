\documentclass[12pt]{amsart}
\usepackage{geometry} % see geometry.pdf on how to lay out the page. There's lots.
\geometry{a4paper} % or letter or a5paper or ... etc

\begin{document}
\begin{enumerate}
\item Study the other examples as listed in pp. 161, and discuss in each case, why it is more natural to use threads. \\
\begin{enumerate}
\item Foreground and background work: This way is more natural with threads than processes because threads are faster at the transferring of information. 
\item Asynchronous processing: If a process were to do this, the program would have to stop every time it went to save. Using a thread lets the main process continue and execute the save discretely.
\item Speed of execution: Threads make more sense here because using processes are very slow. To spawn a new process would take both a multiprocessor and the memory for it.
\item Modular Program Structure: Programs that use multiple threads will run, on average, faster and with more efficiency than a program that uses different processes.\\
\end{enumerate}
\item Problem 4.1\\
A mode switch between threads in the same process requires less work, and memory than threads across processes.\\
\item problem 4.2 \\
The thread(s) are blocked because the operating system doesn't know they are they. This means that the OS can't execute them.\\
\item problem 4.4 \\
This model makes multithreaded programs run faster than their single-threaded counterparts because no processes are blocked. The processor can easily transition from one thread to the next.\\
\item problem 4.5 \\
No, the process exits, and the threads are destroyed.\\
\item Check out the application example of the Valve Game software, and send in a report on why and how a multi-core system is effectively used in this application.\\

In valve's steam client, it must effectively run the store, web browser, and chat client. This means that it must use a threaded process in order to keep the program running quickly, and without hogging all of the system memory. \\
\item Self-study �4.5 on Solaris thread and SMP management.\\
\item Problem 4.7 \\
\begin{enumerate}
\item The program creates a list p. It then cycles though the list, and for every positive value, it increments the global positives.
\item The threads shouldn't execute in parallel on a single-thread execution. They could run into a resource sharing issue. \\
\end{enumerate}
\item Problem 4.8.\\
If A and B are run concurrently, then the global positives could increment without the value being positive.
\end{enumerate}
\end{document}