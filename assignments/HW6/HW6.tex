\documentclass[12pt]{amsart}
\usepackage{geometry} % see geometry.pdf on how to lay out the page. There's lots.
\usepackage{amsmath}
\usepackage{mathtools}
\geometry{a4paper} % or letter or a5paper or ... etc

\begin{document}

Problem 6.1 \\
\begin{enumerate}
\item Mutual Exclusion: Only one car may move at a time.
\item Hold and Wait: Because only one car may move at a time, the other cars must wait. Because no car can go first, they must all wait.
\item Preemption: There is no light to tell which car to go first.
\item Circular Wait: All four cars arrived at the same time, so no car has the right of way.
\end{enumerate}

Problem 6.2 \\ 
\indent Inserting a round-a-bout instead of a four-way-stop will both prevent and avoid the issue. \\

Problem 6.3 \\
\begin{enumerate}
\item Q acquires B and then A and then releases B and A. When P resumes execution, it will be able to acquire both resources
\item Q acquires B and then A. P executes and blocks on a request for A. Q releases B and A. When P resumes execution, it will be able to acquire both resources.
\item Q acquires B and then P acquires A. P then releases A. Q acquires A and then releases B and A. When P resumes, it will be able to acquire both A and B.
\item P acquires A and Q acquires B. P then releases A and Q acquires A and then releases A. P resumes and gets and then releases B.
\item P Acquires A and then B. Q executes and blocks on a request for B. P releases A and B. When Q resumes execution, it will be able to acquire both resources.
\item P acquires A and then B and then releases A and B. When Q resumes execution, it will be able to acquire both resources. \\
\end{enumerate}
Problem 6.4 \\
\indent If a process does not need a resource, it will let go of that resource such that two processes can't fight over it. \\

\newpage

Problem 6.5 \\
\begin{enumerate}
\item True
\item $\begin{matrix}
Process & A & B & C & D \\
P0 & 7 & 5 & 3 & 4\\
P1& 2 & 1 & 2 & 2\\
P2 & 3 & 4 & 4 & 2\\
P3 & 2 & 3 & 3 & 1\\
P4 & 4 & 1 & 2 & 3\\
P5 & 3 & 4 & 3 & 3\\ 
\end{matrix}$ \\
\end{enumerate}

Homework: Self-study �6.6 and discuss in detail how the �rst solution leads to a deadlock, while the second prevents it from happening.\\ 
\indent In the first solution, there is no mutual exclusion, preemption or circular wait. Each philosopher can take more than one fork and they can all take a fork at once. This will lead to dead lock most of the time. The second solution provides that there are an abundance of resources, or that there is a limit on the amount of philosophers (processes) that can be at the table.

\end{document}